\documentclass[a4paper, 11pt]{article}
\usepackage[utf8]{inputenc}
\usepackage[margin=1in]{geometry} 
\usepackage{amsmath}
\usepackage{amsfonts}
\usepackage{amssymb}
\usepackage{graphicx} % Required for inserting images
\usepackage{amsthm}
\usepackage{enumitem}
\usepackage{verbatim}
\usepackage{soul}

\title{Homework 1}
\author{
    Jonathan Arnoult \\ jonathan.arnoult@epfl.ch
    \and Emilien Ganier \\ emilien.ganier@epfl.ch
    \and Marcin Wojnarowski \\ marcin.wojnarowski@epfl.ch
}

\newtheorem{claim}{Claim}

\date{November 22, 2024}
\begin{document}

\maketitle

\section*{Problem 1}

\subsection*{a}
We start by explaining what do the variables refer to. For all $i\leq n$, $y_i=1$ if attraction $x_i$ is visited and $y_i=0$ otherwise. For all $j\leq m$, $z_j=1$ if traveler $j$ is happy and $z_j=0$ otherwise. Each constraint ensures that the values of $y_i$ and $z_j$ are coherent with the definition of happiness.\\\\
We can first make sure that $\sum^m_{j=1} w_jz_j$ corresponds to the revenue of the travel guide. This is true since it exactly includes each happy traveler (i.e. those who come and pay) and add their respective contribution $w_j$, while ignoring unhappy travelers for who $z_j=0$.\\\\
Now we must check if any feasible solution is indeed an itinerary, i.e. if the constraints are properly defined. If $z_j=0$, then we have nothing to check since traveler $T_j$ doesn't come. If $z_j=1$, as $y_i\geq 0$ and $1-y_i\geq 0$ and the problem is integral, there must be at least one $x_i\in V_j$ such that $y_i=1$, or one $x_i\in N_j$ such that $y_i=0$. The first case means a desired attraction is in the itinerary and the traveler will be happy, the second case means a hated attraction is not visited, and the traveler is also happy. In both case the traveler is indeed happy.\\\\
Finally, we need to check that any itinerary satisfies the constraints (and in particular a maximizing itinerary). Let's consider an itinerary and defined $y_i$ and $z_j$ accordingly. If $z_j=0$, then constraint $j$ is satisfied since $y_i\geq 0$ and $1-y_i\geq 0$. If $z_j=1$, then it means the traveler is happy and either a desired attraction is visited or a hated attraction is not. In the first case we have $x_i\in V_j$ such that $y_j=1$ and the constraint is satisfied. In the second case we have $x_i\in N_j$ such that $y_j=0$ and $1-y_j=1$ and the constraint is satisfied. Therefore, the constraints are all satisfied.\\\\
Hence, the integral program finds an itinerary maximizing the revenue.

\subsection*{b}
\subsubsection*{(i)}
We write $Pr(T_j\text{ not happy})$ the probability that traveler $T_j$ is unhappy. We have
\begin{align}
\begin{split}
    Pr(T_j\text{ not happy}) ={}&Pr(\text{all desired attractions are not visited)}\times\\
    & Pr(\text{all hated attractions are visited})
\end{split}\\
\begin{split}
    ={}&\prod_{x_i\in V_j}Pr(\text{attraction } x_i \text{ not visited)}\times\\
    & \prod_{x_i\in N_j}Pr(\text{attraction } x_i \text{ visited})
\end{split}\\
    ={}& \prod_{x_i\in V_j}(1-y_i^*) \prod_{x_i\in N_j}y_i^*\\
    ={}& \prod_{x_i\in V_j\sqcup N_j} a_i
\end{align}
where $a_i=1-y_i^*$ if $x_i\in V_j$ and $a_i=y_i^*$ if $x_i\in N_j$. Here we assume that $V_j$ and $N_j$ are disjoints as it seems reasonable, but this is technically not required for the final inequality.\\\\
Since $a_i\geq 0$, with AM-GM inequality we obtain
\begin{align}
    Pr(T_j\text{ not happy}) \leq{}&\frac{1}{l_j^{l_j}}\left(\sum_{x_i\in V_j\sqcup N_j} a_i\right)^{l_j}\\
    ={}&\frac{1}{l_j^{l_j}}\left(\sum_{x_i\in V_j} (1-y_i^*)+\sum_{x_i\in N_j} y_i^*\right)^{l_j}\\
    ={}&\frac{1}{l_j^{l_j}}\left(\lvert V_j\rvert-\sum_{x_i\in V_j} y_i^*-\sum_{x_i\in N_j} (1-y_i^*)+\lvert N_j\rvert\right)^{l_j}\\
    \leq{}&\frac{1}{l_j^{l_j}}\left(l_j-z_j^*\right)^{l_j}\\
    ={}&\left(1-\frac{z_j^*}{l_j}\right)^{l_j}
\end{align}

For $(8)$, we use the constraints inequalities.

\subsubsection*{(ii)}
We bound the expected profit
\begin{align}
    E (\text{profit}) ={}&\sum_{j=1}^m w_j Pr(T_j\text{ happy})\\
    ={}&\sum_{j=1}^m w_j (1-Pr(T_j\text{ not happy}))\\
    \geq{}& \sum_{j=1}^m w_j \left(1-\left(1-\frac{z_j^*}{l_j}\right)^{l_j}\right)\\
    \geq{}& \sum_{j=1}^m w_j \left(1-\left(1-\frac{1}{l_j}\right)^{l_j}\right)z_j^*\\
    \geq{}& \sum_{j=1}^m w_j \left(1-\frac{1}{e}\right)z_j^*\\
    ={}& \left(1-\frac{1}{e}\right)\sum_{j=1}^m w_j z_j^*\\
    ={}& \left(1-\frac{1}{e}\right)OPT_{LP}\\
    \geq{}& \left(1-\frac{1}{e}\right)OPT_{INTEGRAL}
\end{align}
which concludes. For $(12)$, we use the fact that all $w_j$ are non negative and the previous question. For $(13)$ and $(14)$ we use the suggested inequalities. For $(17)$ we use the fact that the profit of the LP relaxation is always equal or greater than the profit of the integral problem, and $\left(1-\frac{1}{e}\right)>0$.\\\\
Therefore, we have a $\left(1-\frac{1}{e}\right)$-approximation.

\newpage

\section*{Problem 2}
\subsubsection*{Equivalent formulation}
Let $f$ be a feasible solution to the LP problem. We remark that:
$$
\sum_{p \in \mathcal{P}}f_p = \sum_{e \in E}\sum_{p \in \mathcal{P}: e \in p}f_p \le \sum_{e \in E}1\le|E| \le n^2
$$

So an equivalent formulation of the problem is given by:

\begin{align*}
    \text{maximize }&\sum_{p \in \mathcal{P}} f_p \\
    \text{s.t. }&\sum_{p \in \mathcal{P} : e \in p} f_p \le 1 \qquad
\forall e \in E \\
    &\sum_{p \in \mathcal{P}}f_p \le n^2
\end{align*}

\subsubsection*{Algorithm}

We consider the following algorithm:

\begin{itemize}
    \item Assign each edge $e$ a weight $w_e^{(1)}$ initialized to 1.
\end{itemize}
At time $t$: % TODO: define T
\begin{itemize}
    \item Pick the distribution $p_e^{(t)} = \frac{w_e^{(t)}}{\Phi^{(t)}}$ where $\Phi^{(t)} = \sum_{e} w_e^{(t)}$.
    \item Solve the reduced LP problem using an oracle (see below).
    \item Define the cost of constraint $e$ as:
        $$\boxed{m_e^{(t)} = 1 - \sum_{p\in\mathcal{P} : e \in p} f_p^{(t)}}$$
        We note that the cost is positive when the constraint is satisfied.
    \item Update the weights with $w_e^{(t+1)} = w_e^{(t)} e^{-\epsilon m_e^{(t)}}$.
\end{itemize}

Output : the average $\bar{f} = \frac{1}{T}\sum_{t = 1}^{T}f_p^{(t)}$.

\subsubsection*{Reduced LP Problem}

At a given time $t$, the reduced LP problem is:
\begin{align*}
    \text{maximize }&\sum_{p \in \mathcal{P}} f_p^{(t)} \\
    \text{s.t. }&\sum_{p \in \mathcal{P}} \left(\sum_{e \in p} p_e^{(t)}\right)f_p^{(t)} \le 1 \qquad
\forall p \in P, f_p^{(t)} \ge 0 \\
    &\sum_{p \in \mathcal{P}}f_p^{(t)} \le n^2
\end{align*}

Let us consider the same graph as $G$ with weights on edges given by $(p_e^{(t)})_{e\in E}$. Let us denote that graph $G'$.

The reduced LP problem amounts exactly to finding a \ul{shortest path} $p^*$ between $s$ and $t$ in $G'$.

Indeed, an optimal solution to the LP problem is given by $f_{p}^{(t)} = \min\left(\frac{1}{\sum_{e \in p^*} p_e^{(t)}}, n^2\right)$, where $p^*$ is a shortest path in $G'$. The $p_e^{(t)}$ are all positive so this \ul{reduced problem be solved in $O(n^2)$} using \ul{Dijkstra's algorithm}. This is an even simpler case than the fractional knapsack, because we can put all the "weight" on one path only.

\subsubsection*{Bound on cost vectors}

Let us establish the bound on the cost vectors. Let $t$. By positivity of $f_p^{(t)}$ for all $p$, $m_e^{(t)} \le 1$. Since $\sum_{p\in\mathcal{P}}f_p^{(t)} \le n^2$, we get that $m_e(t) \ge 1 - n^2$. So we have a bound $\rho:=max_{t,e}(|m_e^{(t)}|)$ in $poly(n)$ on the cost vectors.

% TODO


\subsubsection*{First requirement}

By the corollary to the Hedge algorithm analysis, if we run the Hedge algorithm on these cost vectors in time $T \ge 4\rho^2\ln{|E|}/\epsilon^2$ we get that for each edge $e$:
$$\frac{1}{T}\sum_{t = 1}^{T}\sum_e p_e^{(t)}m_e^{(t)} \leq \frac{1}{T}\sum_{t=1}^{T}m_e^{(t)}+2\epsilon$$

Let $e \in E$. The left-hand side is precisely:
$$
\frac{1}{T}\sum_{t = 1}^{T}\sum_e p_e^{(t)}m_e^{(t)}
    = \frac{1}{T}\sum_{t = 1}^{T}\sum_e p_e^{(t)}\left(1 - \sum_{p\in\mathcal{P} : e \in p}f_p^{(t)}\right) \\
    = \frac{1}{T}\sum_{t = 1}^{T}\underbrace{1 - \sum_{p \in \mathcal{P}} \left(\sum_{e \in p} p_e^{(t)}\right)f_p^{(t)}}_{\ge 0\text{ , by feasibility of } f_p^{(t)}}$$

Whence the positivity of the left-hand side, and therefore the positivity of the right-hand side. Consequently:

$$-2\epsilon \le \frac{1}{T}\sum_{t = 1}^{T}m_e^{(t)} = 1 - \frac{1}{T}\sum_{t=1}^{T}\sum_{p \in \mathcal{P} : e \in p}f_p^{(t)} = 1 - \sum_{p \in \mathcal{P}:e \in p}\bar{f_p}$$

So for all $e \in E$, \fbox{$\sum_{p:e \in p}\bar{f_p} \le 1 + 2\epsilon$}.

\subsubsection*{Second requirement}

Let $f^*$ be an optimal solution to the max-flow LP. Since $\sum_{e \in p} p_e^{(t)} \le 1$ and $f_p^* \ge 0$ for all $p$, we have:
$$\sum_{p\in \mathcal{P}:e\in p}\left(\sum_{e \in p} p_e^{(t)}\right)f_p^* \le \sum_{p \in \mathcal{P} : e\in p} f_p^* \le 1$$

Also, by our equivalent formulation:
$$
\sum_{p \in \mathcal{P}}f_p \le n^2
$$

Hence, $f^*$ is a feasible solution of the reduced LP problem. Let $t$. By maximality of $(f_p^{(t)})_p$ for the reduced LP problem,
$$\sum_{p \in \mathcal{P}} f_p^{(t)} \ge OPT$$

By averaging the preceding result over $t$:
$$\boxed{\sum_{p \in \mathcal{P}}\bar{f_p} \ge OPT}$$

\subsubsection*{Runtime}

% T but not multiply by oracle

We remark that $\rho = poly(n)$ and $|E|=poly(n)$ so $T = O(4\rho^2\ln{|E|}/\epsilon^2) = O(poly(n)/\epsilon^2)$. Each iteration is in $poly(n)$ since the oracle is in $poly(n)$.

All in all, our algorithm runs in time \fbox{$poly(n)/\epsilon^2$}.

\newpage

\section*{Problem 3}

The solution was uploaded by the user \texttt{shilangyu}.

\end{document}